% Created 2024-10-04 Friday 21:58
% Intended LaTeX compiler: pdflatex
\documentclass[11pt]{article}
\usepackage[utf8]{inputenc}
\usepackage[T1]{fontenc}
\usepackage{graphicx}
\usepackage{longtable}
\usepackage{wrapfig}
\usepackage{rotating}
\usepackage[normalem]{ulem}
\usepackage{amsmath}
\usepackage{amssymb}
\usepackage{capt-of}
\usepackage{hyperref}
\usepackage{fontspec}
\setmainfont{Noto Serif CJK TC}
\setsansfont{Noto Sans CJK TC}
\date{\today}
\title{}
\hypersetup{
 pdfauthor={},
 pdftitle={},
 pdfkeywords={},
 pdfsubject={},
 pdfcreator={Emacs 31.0.50 (Org mode 9.7.9)}, 
 pdflang={English}}
\begin{document}

\begin{titlepage}
  \begin{center}
    \vspace*{1cm} \textbf{说明文档} \vspace{0.5cm} dotfiles配置文件说
    明\vspace{1.5cm} \\ \textbf{une} \vfill dotfiles配置文件说明\vspace{0.8cm}
    \\ 2024-10-04 \\ \LaTeX \\ \vspace{0.6cm} 感谢GNU Emacs 与 AUCTeX \\ 因为这让进行\LaTeX{}的编写变得极为轻松
  \end{center}
\end{titlepage}

\tableofcontents

\section{dotfiles配置文件说明}

\subsection{常用工具}
\begin{center}
  \begin{tabular}{|c|c|}
    \hline
    工具 & 说明 \\
    \hline
    桌面环境 & KDE plasma 6 \\ 
    窗口管理器 & i3wm(X11) \\
    窗口bar条 & eww \\
    壁纸管理器 & feh \\  
    主力编辑器 & GNU Emacs \\
    备用编辑器 & vim \\
    终端模拟器 & alacritty \\
    \hline
  \end{tabular}
\end{center}

\subsection{次要工具}
我的意思是这些工具并不是我个人在过去的半年里常用的。 \\ 但如果您想要参考的话,也
可以参考一下。毕竟我自己也是参考别人的。
\begin{center}
  \begin{tabular}{|c|c|}
    \hline
    工具 & 说明 \\
    \hline  
    桌面服务器 & Wayland \\
    窗口管理器 & Hyprland \\
    所用编辑器 & neovim \\
    终端复用器 & tmux \\ 
    \hline
  \end{tabular}  
\end{center}

\newpage{}
\section{常用工具快捷键说明}
\subsection{i3wm(X11)}
\begin{center}
  \begin{tabular}{|c|l|l|}
    \hline
    \multicolumn{2}{|c|}{个人常用快捷键} & 说明 \\   
    \hline
    Super & R & 重启i3wm \\      
    Super & E & 窗口全屏 \\
    Super & A & 窗口截图(使用KDE截图工具spectacle) \\
    Super & W & 调整窗口(首次按下调整,再次按下完成调整,用FBNP调窗口大小) \\
    Super & I & 打开浏览器(默认Firefox) \\        
    Super & O & 打开终端模拟器alacritty \\  
    Super & RET & 打开终端模拟器alacritty \\
    Super & F & 光标移动到右边窗口 \\
    Super & B & 光标移动到左边窗口 \\
    Super & N & 光标移动到下边窗口 \\
    Super & P & 光标移动到上边窗口 \\
    Super & Meta(Alt) + F & 当前窗口,与右边窗口对调 \\
    Super & Meta(Alt) + B & 当前窗口,与左边窗口对调 \\
    Super & Meta(Alt) + N & 当前窗口,与下边窗口对调 \\
    Super & Meta(Alt) + P & 当前窗口,与上边窗口对调 \\
    \hline
  \end{tabular}  
\end{center}

\newpage{}
\subsection{GNU Emacs}
\begin{center}
  \textbf{其中,C为Ctrl,M为Meta(Alt), S为Shift}
\end{center}
\begin{flushleft}
  个人常用快捷键(需添加此仓库里的Emacs配置才能用的快捷键)
\end{flushleft}
\begin{left}
  \begin{tabular}{|c|l|l|}
    \hline
    \multicolumn{2}{|c|}{个人常用快捷键} & 说明 \\ 
    \hline
    C-c & v & 进入选区模式 \\
    C-c & g o & 打开Magit \\
    C-c & t o o o & 打开ansi-term \\
    C-c & e o & 打开eglot \\
    C-c & f & 光标跳转到右边窗口 \\
    C-c & b & 光标跳转到左边窗口 \\
    C-c & n & 光标跳转到下边窗口 \\
    C-c & p & 光标跳转到上边窗口 \\
    C-c & k & 删除当前光标所在窗口 \\
    C-c & d o & 打开dirvish \\
    M & n & 当前光标向下5行 \\
    M & p & 当前光标向上5行 \\
    M & 9 & 当前光标所在窗口水平缩小 \\
    M & 0 & 当前光标所在窗口水平扩大 \\
    M & - & 当前光标所在窗口垂直缩小 \\
    M & + & 当前光标所在窗口垂直扩大 \\            
    \hline
  \end{tabular}  
\end{left}
\newpage{}
\begin{flushleft}
  默认Emacs快捷键(直接就能用的快捷键)
\end{flushleft}
\begin{left}
  \begin{tabular}{|c|l|l|}
    \hline
    \multicolumn{2}{|c|}{默认Emacs快捷键} & 说明 \\
    \hline
    \textbf{C-h} & \textbf{b} & \textbf{查看当前所属mode的快捷键用法(最好的了解某个mode的最佳实践用法)} \\
    \textbf{C-h} & \textbf{v} & \textbf{查看当前GNU Emacs(lisp机)的某个变量具体运行情况} \\    
    C & f & 光标向右1格位置 \\ 
    C & b & 光标向左1格位置 \\ 
    C & n & 光标向下1格位置 \\ 
    C & p & 光标向上1格位置 \\
    C & a & 光标移动到当前行line开头 \\
    C & e & 光标移动到当前行line结尾 \\
    M & f & 光标向右1个词义位置 \\
    M & b & 光标向左1个词义位置 \\
    M & n & 光标向下1个词义位置 \\
    M & p & 光标向上1个词义位置 \\
    M & a & 光标移动到当前自然段paragraph开头 \\
    M & e & 光标移动到当前自然段paragraph结尾 \\    
    M & w & 进入选区模式后,按下来复制文本 \\
    C & w & 进入选区模式后,按下来剪贴文本 \\
    C & y & 在复制文本或剪贴文本后,按下来粘贴文本 \\
    M & y & 打开kill-ring,并在交互式菜单里使用C-n与C-p选择要粘贴的文本 \\
    C & d & 向后删除光标所在的1个字符 \\
    C & k & 向后删除光标当前行line所有Buffer(并放进kill-ring中) \\
    M & k & 向后删除光标当前自然段paragraph所有Buffer(并放进kill-ring中) \\
    C & j & 向下开辟一行文本并移动光标,同时尽可能满足缩进 \\
    C & BS & 向前删除光标所在的1个词义(并放进kill-ring中,仅在Emacs GUI可用) \\    
    M-S & , & 光标跳转到最上方 \\
    M-S & . & 光标跳转到最下方 \\
    C-x & 1 & 只保留当前光标所在窗口并删除其他窗口 \\
    C-x & 2 & 当前窗口向下分屏 \\
    C-x & 3 & 当前窗口向右分屏 \\
    C-x & k & 删除当前光标所属Buffer \\
    C-x & n n & 当前编辑窗口narrow来提高专注力(narrow与widen) \\
    C-x & n w & 当前编辑窗口widen来还原显示原来的文本(narrow与widen) \\
    C-x & 8 RET & 交互式菜单输入特殊字符(就像vim的C-k一样) \\
    \hline    
  \end{tabular}  
\end{left}

\newpage{}
\subsection{vim}
\begin{flushleft}
  vim个人常用快捷键(\textbf{未完待续})
\end{flushleft}
\begin{left}
  \begin{tabular}{|c|l|l|}
    \hline
    \multicolumn{2}{|c|}{个人常用快捷键} & 说明 \\ 
    \hline
    
    \hline
  \end{tabular}  
\end{left}

\end{document}

