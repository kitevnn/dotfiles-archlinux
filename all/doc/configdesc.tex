\documentclass[11pt]{article}
\usepackage[utf8]{inputenc}
\usepackage[T1]{fontenc}
\usepackage{graphicx}
\usepackage{longtable}
\usepackage{wrapfig}
\usepackage{rotating}
\usepackage[normalem]{ulem}
\usepackage{amsmath}
\usepackage{amssymb}
\usepackage{capt-of}
\usepackage{hyperref}
\usepackage{fontspec}
\setmainfont{Noto Serif CJK TC}
\setsansfont{Noto Sans CJK TC}
\date{\today}
\title{}
\hypersetup{
 pdfauthor={},
 pdftitle={},
 pdfkeywords={},
 pdfsubject={},
 pdfcreator={Emacs 31.0.50 (Org mode 9.7.11)}, 
 pdflang={English}}

% 使用 code 环境给文本上底色
\usepackage{listings, color}
\definecolor{verbgray}{gray}{0.9}
\lstnewenvironment{code}{%
  \lstset{backgroundcolor=\color{verbgray},
    frame=single,
    framerule=0pt,
    basicstyle=\ttfamily,
    columns=fullflexible}}{}
\definecolor{shadecolor}{rgb}{.9,.9,.9}

\begin{document}

\begin{titlepage}
  \begin{center}
    \vspace*{1cm} \textbf{说明文档} \vspace{0.5cm} : 安装要注意的地
    方\vspace{1.5cm} \\ \textbf{une} \vfill 安装要注意的地方\vspace{0.8cm} \\
    \date{\today} \\ \LaTeX \\ \vspace{0.6cm} 感谢GNU Emacs 与 AUCTeX \\ 因为这让
    进行\LaTeX{}的编写变得极为轻松
  \end{center}
\end{titlepage}

\tableofcontents

\newpage{}
\section{环境说明}
此配置环境均在X11桌面服务器下运行
\section{工具安装}
\subsection{eww}
由于eww作者建议说
\begin{code}
  Rather than with your system package manager
  I strongly recommend installing it using rustup.
\end{code}
加之作者勤快的更新速度,因此还是建议自己编译
\\ \\
\textbf{那么编译困难吗?} \\
\textbf{答: 不难,只要使用的发行版是最新版archlinux,就一定有编译出eww的依赖库}
\\ \\
我们先去下载Rustup
\begin{code}
  curl --proto '=https' --tlsv1.2 -sSf https://sh.rustup.rs | sh
\end{code}
\\ \\ 
然后克隆作者的仓库并进行编译
\begin{code}
  git clone https://github.com/elkowar/eww cd eww
  cargo build --release --no-default-features --features x11
\end{code}
\\ \\ 
最后设置软连接
\begin{code}
  cd target/release
  chomd +x ./eww
  sudo ln -s ./eww /usr/bin/eww
\end{code}
\\ \\

\newpage{}

\end{document}
